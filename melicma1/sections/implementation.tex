\chapter{Implementace}
    Při vytváření kódu bylo nutné dodržovat domluvené návrhy. Kód musel být přehledný pro případné rozšířění aplikace jiným vývojářem. Díky vybranému softwaru a jeho přívětivé příručce byla poměrně rychle vytvořena aplikace v základní podobě. Slim má obrovskou komunitu uživatelů, což je velice nápomocné při hledání řešení problému. 

    \section{Import}
        Ještě před implementací se musely do databáze aplikace zpracovat sbírky z UČL AV. Podle očekávání byla data dodána ve formátu XML. Import nedoprovázely žádné komplikace zejména díky snadné ovladatelnosti a standardním knihovnám scriptovacího jazyka Python\footnote{domovská stránka: \url{https://www.python.org//}}. Současně s importem sbírek proběhlo zpracování příloh k nim přidružených.
        
        \subsection{Stávající sbírka děl}
            Stávající díla mají pracovníci z UČL AV v databázi. Část z této sbírky děl je ústavem již zpracována, na zbylé části se stále pracuje. Dříve byla tato díla upravována pomocí kancelářského balíčku Microsoft Word\footnote{domovská stránka: \url{https://products.office.com/cs-cz/word}}. Word je skvělý nástroj pro editaci jednotlivých souborů, nicméně nedovoluje spravovat soubory jako celek. V současné době pracovníci UČL AV čekají na novou aplikaci, protože jim chybí vhodný nástroj k úpravě děl.
            
            Import literárních děl proběhl otevřením každého souboru a zpracováním skriptem. Skript získal informace o dílu, vytvořil nové nebo navázal na již existující autory či vydavatele a vytvořil záznamy v tabulkách. Pro práci s XML byla použita standardní knihovna ElementTree\footnote{domovská stránka: \url{https://docs.python.org/2/}}. Python má stejnojmenou knihovnu pro práci s databází sqlite.
            
            V~ElementTree se na začátku zavolá funkce getroot. Tato funkce inicializuje proměnnou, ve které je obsah souboru a  dá se v ní snadno přistupovat k jednotlivým tagům. Tag hlavicka indikuje údaje o dílu a obsah díla patří pod tag text. Funkce find slouží k nalezení a vrácení tagu podle jeho názvu.
            
            Fulltextové vyhledávání v aplikaci vyžaduje text bez tagů. K tomu slouží funkce itertext, která ignoruje tagy a vrátí celý text. Naopak funkce pro získání obsahu včetně tagů není v ElementTree vestavěna. Nicméně potřebný text lze dostat spojením jiných funkcí. Obsah této funkce byl inpirován \cite{fn01}. 
            
            Některá díla mají více autorů. V takovém případě všechna jména zapsána v tagu author. Tato jména jsou ohraničena množinovými závorkami a oddělena středníkem. Mnoho autorů si v průběhu své tvorby vymyslelo pseudonym. Pseudonym autora je indikován znakem \uv{=}. Vlevo od znaku je pseudonym a vpravo je pravé jméno autora. Funkce doAuthors vrací id autora nebo pole id autorů v závislosti na vstupním parametru authors. Tento parametr je typu string a pokud začíná znakem \uv{\{}, má dílo více autorů. Pro tento případ funguje funkce jako rekurze. Druhým parametrem funkce je indikátor rekurze (recursion).
            
            Pro ilustraci je uveden začátek funkce zpracování autorů díla.
             \begin{minted}[frame=single, autogobble=true, label=doAuthors]{python}
def doAuthors(authors, recursion):
    #there is no author
    if (authors == 'neznamy'):
        return -1
    #there are more authors
    if (authors[0] == '{'):
        authors = authors.replace('{', '')
        authors = authors.replace('}', '')
        authArr = authors.split(';')
        authArr[0] = ' ' + authArr[0]
        out = []
        for authorName in authArr:
            out.append(doAuthors(authorName, 1))
        return out
    #there is only author 
    elif (authors.find('=') == -1):
        authors = authors.replace('(', '')
        authors = authors.replace(')', '')
        lastName, sep, name = authors.partition(',')
        name = name[1:]
        if (recursion == 1):
            lastName = lastName[1:]
        #create or find and select author id from DB    
        return getAuthorId(name, lastName)
             \end{minted}
            
            Pro obsluhu vydavatelů byla použita funkce doPublisher, která pracuje podobně jako funkce pro autory, nicméně se zde nevyskytují pseudonymy. 
            
            Import do tabulky works proběhl ve funkci doWorks. Funkce má v parametrech název díla (title), rok vydání (year), status, proměnnou, která obsahuje ostatní informace (meta), fulltext, text (content) a poznámku k vydání (note). Funkce vrací id vytvořeného díla. Příklad zjištění počtu stránek díla z parametru meta:
            \begin{lstlisting}
    pages = ''.join(meta.find('stran').itertext())
            \end{lstlisting}
            
            Záznamů do tabulky connection byly vloženy pomocí funkce doConnection. Tato funkce má parametry workId, indexId a typeOfConn. První je id vytvořeného díla a indexId je pole id autorů nebo vydavatelů, které může obsahovat jeden prvek. TypeOfConn je typ propojení, kterým je buď author nebo publisher. Obsah funkce je pouze databázový dotaz typu insert.
            
        \subsection{Přílohy}
            Zaměstnanci UČL AV dodali společně s některými díly i jejich naskenované stránky. Adresář scan obsahoval  podadresáře img, kde byly hlavní obrázky a thumbs, kde se nacházely zmenšené obrázky. Jednotlivá díla jsou nazvána číslem například 0001.xml. Tato čísla byla zároveň adresářem v img i thumbs. Podle čísla se zjistilo, ke kterému dílu obrázky patří. V číselných podadresářích byly uloženy obrázky ve formátu jpg a u hlavních obrázků byl soubor pages.csv, ve kterém byly informace o jednotlivých stránkách. Přílohy byly uloženy s číselným názvem. Příklad záznamu v souboru pages: 002.jpg;Strana [1]. Tuto informaci bylo nutné zpracovat a vložit do tabulky jako poznámku k příloze. Problém nastal při otevírání soboru pages.cvs, protože pracovníci si nepamatovali v jakém kódování soubor uložili. Po chvilkovém bádání se zjistilo, že použili cp1250. 
            
            V aplikaci se přílohy přidávají do adresáře images a podadresáře nazvaném podle id díla. §§K tomuto id byly kvůli řazení přidány nuly z leva, aby mělo 5 míst.§§ Názvy příloh jsou u každého díla tvořena číslem začínajícím od 1. Pro snadnější řazení se jméno skládá ze tří znaků, čísla a nezbytné nuly na začátku. Zmenšené obrázky mají ke jménu přidanou příponu \_small, příklad názvu obrázku: 010\_small.jpg. Vytvářet adresáře lze pomocí knihovny os, čtení souboru zajistí knihovna pandas a obsluhu souborového systému obstará knihovna shutil. Pro import příloh byla použita následující funkce:

%\pagebreak
            \begin{minted}[frame=single, autogobble=true, label=doAttachments]{python}
def doAttachments(workID, oldID):
    #create directory if not exists
    os.makedirs('./images/' + '{0:0=5d}'.format(workID),
                exist_ok=True)
    tmp = './scan/scan/img/'+str(oldID)+'/'
    tmpSmall = './scan/scan/thumbs/'+str(oldID)+'/'
    khe = pd.read_csv(tmp + 'pages.csv',
                      encoding='cp1250',
                      sep=';', header=None)
    khe.columns = ['id', 'poznamka']
    l = 1
    for value in khe.id.keys():
        db.execute(insertAttachmentSQL,
                   [workID, khe.poznamka[value],
                   str('{0:0=3d}'.format(l)) + '.jpg'])
        sh.copy(tmp + khe.id[value], 
               './images/' + 
                str('{0:0=5d}'.format(workID)) + '/' +
                str('{0:0=3d}'.format(l)) + '.jpg')
        sh.copy(tmpSmall + khe.id[value], 
               './images/' + 
               str('{0:0=5d}'.format(workID)) + '/' +
               str('{0:0=3d}'.format(l)) + '_small.jpg')
        l = l + 1
            \end{minted}

            
            
    \section{Aplikace}
    
        \subsection{Kostra aplikace}
            základní aplikace pomoci slimframeworku
            
        \subsection{Login}
            popis login funkcí
            
        \subsection{Seznam děl}

        \subsection{Seznam autorů a vydavatelů}
        
        \subsection{Metadata}
        
        \subsection{Přidat nového autora nebo vydavatele}
        
        \subsection{Upravit autora nebo vydavatele}

        \subsection{Přílohy}



