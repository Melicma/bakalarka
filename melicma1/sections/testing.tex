
\chapter{Testování}
        Testování aplikace je důležitou součástí projektu, protože se ještě před vydáním aplikace zjistí, zda funguje vše, jak má. Díky testování se dá předejít případným fatálním chybám systému. Pro testování z~hlediska kódu byla výbrána metoda jednotkových testů a pro testování, zda je prostředí uživatelsky přívětivé, se použila metoda kognitivního průchodu aplikace.
        
    \section{Uživatelské testování}
        Uživatelské testování probíhalo formou kognitivních průchodů, protože jsem se s~nimy setkal v~průběhu studia. Kognitivní průchod je prediktivní metoda testování aplikace. Díky této metodě se zjistí, jestli je aplikace přehledná a zda se dá ovládat intuitivně. K~provedení testu jsou potřeba alespoň tři respondenti, testovací scénáře a testovací artefakt. 
        
        Relevantní výsledky testu zajistí výběr osob z~různých věkových katogorií a odlišného zaměstnání. Prvním subjektem je studentka VŠ ve věku 23 let, která před rokem dosáhla bakalářského stupně vzdělání. Počítač používá denně okolo 6 hodin zejména k~aktivitám na sociálních sítích. Druhým testerem je pětatřicetiletý dělník, který má základní vzdělání a denně stráví na počítači okolo 2 hodin. Poslední účastník testu je prodavač, který dovršil 53 let. Jeho dosažené vzdělání je střední s~maturitou a denně stráví méně jak 3 hodiny na počítači.
        
        \subsection{První scénář}
            Prvním úkolem pro účastníky bylo přidání pseudonymu k~autorovi Petr Bezruč. Na tomto scénáři se testovalo, zda uživatelé pochopí, že musí založit nového autora a přiřadit referenci k~reálné osobě.
            
            Při plnění zadání respondentům nejdéle trvalo najít, kde se přidává nový záznam autora. Poté se stalo, že testeři hledali samotný záznam Petr Bezruč a zkoušeli jej upravit. Nicméně po menší radě se povedlo všem přidat pseudonym.
            
            Díky připomínkám osob se zvýraznil odkaz v~menu na seznam autorů a vydavatelů a změnila se pozice tlačítka pro založení nového záznamu. Nově je tlačítko viditelné ihned po přesměrování na seznam autorů a vydavatelů. 

        \subsection{Druhý scénář}
            Jako druhá akce, kterou měli testeři projít, bylo nahrazení poslední přílohy u~díla \uv{České zpěvy}. Tento úkol měl ověřit, jak rychle dokážou uživatelé nalézt příslušné dílo, jestli dokážou smazat poslední přílohu a zda nahrají novou přílohu bez problémů.
            
            Pro vyhledání díla použili respondenti filtr v~horní části stránky i vestavěný filtr v~tabulce děl. Intuitivně ihned nalezli poslední přílohu. Nicméně ze začátku zkoušeli nahradit původní přílohu novou. Až po chvíli pochopili, že musí nejprve přílohu smazat a poté nahrát novou.

            Výsledkem testu je posunutí prvku pro nahrání příloh výš, aby bylo lépe vidět, a vložení nápovědy. Tato nápověda se zobrazí po najetí kurzoru na otazníček v~kruhu.

        \subsection{Třetí scénář}
            Posledním scénářem bylo upravení ediční poznámky díla, jejímž autorem je Ferdinand Tomek. Zde se kontrolovalo vyhledávání v~dílech a zda je formulář pro změnu údajů o~dílech snadno pochopitelný.
            
            V~průběhu testování využily osoby zkušenosti s~vyhledáváním z~předešlých scénářů a neměly žádný problém s~úpravou záznamu. 
            
            Po opakovaném použití aplikace respondetni pochvalují jednoduchost a snadné ovládání aplikace. Po zpětné vazbě byly opraveny nejasné pasáže, zvýrazněny hlavní funkce a přidány krátké nápovědy.

    \section{Jednotkové testy}
        Jednotkové testy prověřují funkčnost jednotlivých částí programu, v~tomto případě funkcí v~souboru \textit{routes.php}. Byly otestovány všechny hlavní stránky od zobrazení literárních děl v~tabulce po úpravu jednotlivých autorů.
        
        Framework Slim dovoluje provádět testování pomocí knihovny PHPUnit. Pomocí této knihovny lze simulovat server aplikace a posílat na něj požadavky. V~odpovědích se testuje, zda odpověď obsahuje očekávaný html status a jestli se renderuje správná šablona. Test správnosti v~odpovědi se provádí pomocí assertů. Příklad jednoho testu včetně assertu.
        
        \pagebreak
        
        \begin{minted}[frame=single, autogobble=true, label=Test změny názvu díla s~WorkID 52]{php}
<?php
public function testPostMetadata()
    {
        $response = $this->runApp('POST', '/metadata/52', 
                                  array('title' => 'test' ));
        $this->assertEquals(302, $response->getStatusCode());
    }
?>
        \end{minted}
        
        Díky jednotkovým testům byly zjištěny menší chyby, které vznikly při aktualizaci databáze, nicméně byly rychle opraveny. Ostatní testy proběhly bez chyb.
        
    \section{Testování ústavem}
        V~průběhu implementace probíhaly konzultace s~ústavem ohledně nejasností vstupních dat. Pracovníci ústavu si často nepamatovali, jak se data ukládají a co přesně znamenají. Například u~některých děl v tagu \textit{autor} nebyl uveden autor díla, nýbrž jeho pseudonym, což bylo pracovníky sděleno až později. Z~toho důvodu bylo obtížné zjistit, co pracovníci ústavu požadují.
        
        Zaměstnanci ústavu nejsou spokojeni s~rychlostí úpravy textu, nicméně ostatní prvky aplikace fungují podle jejich představ. Z~důvodu nedostatku času před termínem odevzdání práce nezbýval prostor pro zrychlení úpravy textu.
