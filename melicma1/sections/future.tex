
\chapter{Budoucí rozšíření}
        
    \section{Responzivita}
        Další výhoda Bootstrapu je rychlé a snadné nastavení mobilního vzhledu. Požadovaného výsledku je možné dosáhnout použitím specialních tříd do prvků html. Tyto třídy určují, jak se má prvek zobrazit na menší obrazovce. V~případě rozhodnutí vytvořit responzivní aplikaci ze strany ústavu postačí přidat výše uvedené třídy a základní mobilní vzhled je hotový.
        
        
    \section{Zobrazení děl}
        V~průběhu vývoje pracovníci ústavu navrhli doplnit zadání o~náhled díla. Jednalo by se o~zobrazení díla při jeho úpravě. Byl by to zajímavý prvek aplikace, nicméně tato vlastnost byla diskutována společně s~vedoucím a bylo rozhodnuto z~časových důvodů nezahrnout požadavek do této bakalářské práce.
        
        Do budoucna se počítá se zpřístupněním děl široké veřejnosti. Pro tento krok by muselo být implementováno rozhraní, kde se budou elektronická díla zobrazovat tak, jak je zaměstnanci upravili. Buďto by se uživatelé přihlašovali do stejné aplikace jako UČL AV a měli by pomocí uživatelské role odepřeny některé funkce, nebo se vyvine úplně nová aplikace, která by se věnovala čistě zobrazení literárních elektronických děl ze stejné databáze jako tato aplikace. V~součastosti preferuje UČL AV druhý scénář.
        
        
