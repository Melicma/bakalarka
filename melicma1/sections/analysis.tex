\chapter{Analýza}

    \section{Existující řešení}
        Před návrhem a implementací aplikace bylo potřeba řádně prozkoumat existující řešení problému, technologií k tvorbě a elektronické formáty, které se užívají pro uchování elektronických děl. Vzhledem k roztoucí poptávce po elektronických děl na úkor papírových se podle očekávání objevilo spoustu aplikací pro editaci těchto děl.\\
        Podle serveru \cite{tei-wiki} pět editorů
        \begin{itemize}
            \item Emeditor
            \item Editix
            \item EditPad Pro
            \item Essential XML Editor 
            \item Exchanger XML Editor
        \end{itemize}

        \subsection{Emeditor}
            Emeditor\footnote{domovská stránka: \url{https://www.emeditor.com/}} patří mezi nejlepší XML editory. Tento software je hlavním produktem americké firmy Emurasoft, Inc. sídlící v Redmondu ve Washingtonu. Firma se nadále stará o podporu i vývoj. Nicméně autorem editoru je Yutaka Emura. Samotná aplikace již vyhrála 24 mezinárodních cen v kategoriích nejlepší webový nástroj nebo nejlepší aplikace roku 2008. 
            
            \subsubsection{výhody}
                \begin{itemize}
                    \item poslední release v17.5.0 vyšla 27.\,února\,2018
                    \item podpora velkých souborů
                    \item použití více jader při větší zátěži
                    \item kódování UTF-8
                    \item konfigurovatelná kontrola pravopisu
                \end{itemize}
                
            \subsubsection{nevýhody}
                \begin{itemize}
                    \item aplikace je placená, ale nabízí trial verzi na 30 dní
                    \item existuje free verze, nicméně v ní chybí zásadní funkce
                    \item editor je pouze pro Windows
                    \item chybí přehledný průvodce základních funkcí po prvním spuštění
                \end{itemize}
                
        \subsection{Editix} 
            Editix\footnote{domovská stránka: \url{http://www.editix.com/index.html/}} je produktem francouzské společnosti JAPISoft SARL. Editor vytvořil Alexandre Brillant. Systém je velice přehledný a intuitivní. Na trhu systém je od roku 2004 a nejnovější verze je EditiX XML Editor 2017 v15. Zákaznící, kteří využívají tento software jsou převážně vzdělávací instituty od University of Oxford po University of Arizona.
            
            \subsubsection{výhody}
                \begin{itemize}
                    \item přehledný program
                    \item aplikace je pro platformy Windows, Linux, MacOS	
                    \item existuje EditiX Community Edition, která je zadarmo
                    \item mnoho užitečných funkcí např. Find and Replace
                    \item obsahuje inteligentní našeptávač, který pomáhá uživatelům
                \end{itemize}
                
            \subsubsection{nevýhody}
                \begin{itemize}
                    \item verze pro je placená, ale nabízí trial verzi na 30 dní
                    \item existuje lite verze, které chybí spoustu funkcí
                    \item zaplacení licence se vztahuje pouze na jednoho uživatele
                    \item první update je zdarma, další se musí zaplatit
                \end{itemize}
                
        \subsection{EditPad Pro}
            EditPad Pro\footnote{domovská stránka: \url{http://www.editpadpro.com/}} je výhradně textový editor, který lze použít například pro HTML, Javascript nebo XML. Software podporuje změnu jazyka napřiklad do francouzštiny, němčiny, polštiny nebo švédštiny. Projeck Just Great Software, pod kterým byla vyvynuta tato aplikace vznikl v roce 1996. Autorem projektu je Jan Goyvaerts, který je zároveň hlavním ředitelem vývojářu projektu.

            \subsubsection{výhody}
                \begin{itemize}
                    \item program je obecný textový editor
                    \item obarvená syntaxe
                    \item dobře pracuje s velkými soubory
                    \item podpora UTF-8
                    \item existuje EditPad Lite verze, která je zdarma
                \end{itemize}
                
            \subsubsection{nevýhody}
                \begin{itemize}
                    \item nemá explicitní podporu pro XML
                    \item aplikace je pouze pro Windows
                    \item chybí vyhledávání v souborech
                    \item verze Pro je placená, ale existuje verze Lite
                \end{itemize}
                
        \subsection{Essential XML Editor}
            Essential XML Editor\footnote{domovská stránka: \url{http://www.philo.de/xmledit/}} je jednoduchý XML editor. Jeho klíčovou vlastnostní je vestavěný XML validátor. Vývojáři dříve pojmenovali program Open XML Editor, ale po zavedení poplatku za některé funkce projekt přejmenovali. Autorem je Dieter Köhler. 
            
            \subsubsection{výhody}
                \begin{itemize}
                    \item program pracuje jako textový editor
                    \item možnost rychle zjistit zda je soubor validní
                    \item trial verze není časově omezená
                    \item vstupní soubor může být v různém kódování
                    \item klávesová zkratka pro každý příkaz
                \end{itemize}
                
            \subsubsection{nevýhody}
                \begin{itemize}
                    \item výstupní soubor pouze v UTF-8 kódování
                    \item aplikace je pouze pro Windows
                    \item pro zpřístupnění některých funkcí nutnost zakoupit klíč
                    \item poměrně zastaralý desing aplikace
                \end{itemize}

        \subsection{Exchanger XML Editor}
            Exchanger XML Editor\footnote{domovská stránka: \url{http://www.exchangerxml.com/editor/}} je určen pro snadnou editaci, prohlížení, správu a konverzi XML souborů. Exchanger pomáhá svojí širokou nabídkou funkcí XML autorům a vývojářům.  Software je prouktem firmy Cladonia, která se zaměřuje na vývoj XML aplikací. 
            
            \subsubsection{výhody}
                \begin{itemize}
                    \item nabízí full verzi na 30 dní
                    \item aplikace je dotupná na všech platformách
                    \item možnost zobrazení základního náhledu
                    \item poskytuje podporu pro TEI formou stáhnutí balíčku
                    \item automatická kotrola, jestli je soubor validní
                \end{itemize}
                
            \subsubsection{nevýhody}
                \begin{itemize}
                    \item při instalaci nutnost najít cestu k JRE manuálně
                    \item zastaralý software
                    \item časově neomezenou verzi je nutno zakoupit
                    \item poslední update proběhl v roce 2010
                \end{itemize}
                
    \section{Stávající aplikace}
        kdo ji vytvořil, kdy. v čem běží
        
        \subsection{Aktuální stav}
            par let už není udržovaná, popsat její grafický vzhled
            
        \subsection{Nedostatky}
            přiklad, nefunguje statistika a je tam až přiliš zbytečností
            
    \section{Technologie}
        základní údaje, php protoze servery na učl
        \subsection{Jazyk}
            v jakém jazyce, proč jsem se rozhodl psát
        \subsection{Elektronický formát}
            jaký elektronický formát použiji pro uchování děl
    \section{Software}
        jaký framework jsem vybral
    \section{Grafické nástroje}
        podle jakých nástrojů jsem dělal grafické návrhy
