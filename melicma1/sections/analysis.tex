\chapter{Analýza}

    \section{Existující řešení}
        Před návrhem a implementací aplikace bylo potřeba řádně prozkoumat existující řešení problému, technologií k tvorbě a elektronické formáty, které se užívají pro uchování elektronických děl. Vzhledem k roztoucí poptávce po elektronických děl na úkor papírových se podle očekávání objevilo spoustu aplikací pro editaci těchto děl.\\
        Podle serveru \cite{tei-wiki} pět editorů
        \begin{itemize}
            \item Emeditor
            \item Editix
            \item EditPad Pro
            \item Essential XML Editor 
            \item Exchanger XML Editor
        \end{itemize}

        \subsection{ANGLES}
            ANGLES\footnote{domovská stránka: \url{https://www.emeditor.com/}} patří mezi nejlepší XML editory. Tento software je hlavním produktem americké firmy Emurasoft, Inc. sídlící v Redmondu ve Washingtonu. Firma se nadále stará o podporu i vývoj. Nicméně autorem editoru je Yutaka Emura. Samotná aplikace již vyhrála 24 mezinárodních cen v kategoriích nejlepší webový nástroj nebo nejlepší aplikace roku 2008. 
            
            \subsubsection{výhody}
                \begin{itemize}
                    \item poslední release v17.5.0 vyšla 27.\,února\,2018
                    \item podpora velkých souborů
                    \item použití více jader při větší zátěži
                    \item kódování UTF-8
                    \item konfigurovatelná kontrola pravopisu
                \end{itemize}
                
            \subsubsection{nevýhody}
                \begin{itemize}
                    \item aplikace je placená, ale nabízí trial verzi na 30 dní
                    \item existuje free verze, nicméně v ní chybí zásadní funkce
                    \item editor je pouze pro windows
                    \item chybí přehledný průvodce základních funkcí po prvním spuštění
                \end{itemize}
                
        \subsection{Editix} 
            Editix\footnote{domovská stránka: \url{http://www.editix.com/index.html/}} 
            \subsubsection{výhody}
                \begin{itemize}
                    \item 
                    \item 
                    \item 
                    \item 
                    \item 
                \end{itemize}
                
            \subsubsection{nevýhody}
                \begin{itemize}
                    \item aplikace je placená, ale nabízí trial verzi na 30 dní
                    \item existuje free verze
                    \item 
                    \item 
                \end{itemize}
    \section{Stávající aplikace}
        kdo ji vytvořil, kdy. v čem běží
        
        \subsection{Aktuální stav}
            par let už není udržovaná, popsat její grafický vzhled
            
        \subsection{Nedostatky}
            přiklad, nefunguje statistika a je tam až přiliš zbytečností
            
    \section{Technologie}
        základní údaje, php protoze servery na učl
        \subsection{Jazyk}
            v jakém jazyce, proč jsem se rozhodl psát
        \subsection{Elektronický formát}
            jaký elektronický formát použiji pro uchování děl
    \section{Software}
        jaký framework jsem vybral
    \section{Grafické nástroje}
        podle jakých nástrojů jsem dělal grafické návrhy
