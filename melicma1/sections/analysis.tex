\chapter{Analýza}

    \section{Existující řešení}
        Před návrhem a implementací aplikace bylo potřeba řádně prozkoumat existující řešení problému, technologií k tvorbě a elektronické formáty, které se užívají pro uchování elektronických děl. Vzhledem k roztoucí poptávce po elektronických děl na úkor papírových se podle očekávání objevilo spoustu aplikací na správu těchto děl.\\
        Podle serveru \cite{bookrunch} pět nejlepších organizérů
        \begin{itemize}
            \item Calibre
            \item Alfa Ebooks Manager
            \item Delicious Library
            \item LibraryThing
            \item Kindlian
        \end{itemize}

        \subsection{Calibre}
            Calibre\footnote{domovská stránka: \url{https://calibre-ebook.com/}} patří mezi nejlepší systémy svého účelu. Obsahuje mnoho funkcí, které usnadní uživateli práci. Například vyhledávání díla podle tagů nebo možnost doplnění chybějících metadat online. Díky své volné dostupnosti a dlouholetému působení na trhu, systém získal početnou základnu uživatelů. Tito uživatelé pravidelné přispívají k vývojí aplikace peněžně, svými návrhy nebo hlášením nefunkčních prvků.
            
            \subsubsection{výhody}
                \begin{itemize}
                    \item volně dostupný systém
                    \item funguje již od roku 2006
                    \item mnoho vývojářu a testerů udržuje aplikaci
                    \item systém umožňuje stáhnout denní zprávy a konvertovat je do elektronické podoby
                    \item systém umožňuje uživateli zobrazit dílo
                \end{itemize}
                
            \subsubsection{nevýhody}
                \begin{itemize}
                    \item někteří uživatelé mají problémy se zobrazením díla na monitorech
                    \item prostředí komplikováno velkým počtem funkcí
                    \item chybí přehledný průvodce po prvním spuštění
                \end{itemize}
    \section{Stávající aplikace}
        kdo ji vytvořil, kdy. v čem běží
        
        \subsection{Aktuální stav}
            par let už není udržovaná, popsat její grafický vzhled
            
        \subsection{Nedostatky}
            přiklad, nefunguje statistika a je tam až přiliš zbytečností
            
    \section{Technologie}
        základní údaje, php protoze servery na učl
        \subsection{Jazyk}
            v jakém jazyce, proč jsem se rozhodl psát
        \subsection{Elektronický formát}
            jaký elektronický formát použiji pro uchování děl
    \section{Software}
        jaký framework jsem vybral
    \section{Grafické nástroje}
        podle jakých nástrojů jsem dělal grafické návrhy
